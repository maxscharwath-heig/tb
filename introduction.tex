\section{Contexte}
La startup Condensation est une entreprise spécialisée dans le développement de solutions logicielles pour la synchronisation de données distribuées. La bibliothèque de Condensation offre une approche innovante pour garantir la cohérence des données à travers un réseau de nœuds, en utilisant un système de synchronisation conflict-free qui garantit l'intégrité des données et la sécurité de bout en bout.

Le développement du coeur de Condensation a été initié par le professeur \textbf{Thomas Lochmatter} de l'Ecole Polytechnique Fédérale de Lausanne (EPFL), et est actuellement maintenu et développé par la startup Condensation.

Ce projet est intéressant, car la synchronisation de données distribuées est un enjeu majeur dans le domaine de l'informatique, et les solutions existantes ne sont pas toujours satisfaisantes.

Le projet sur lequel porte ce rapport a pour objectif de contribuer au développement de la bibliothèque de Condensation en développant un démonstrateur permettant de mettre en évidence les fonctionnalités du système et d'offrir aux développeurs des exemples concrets pour les aider à intégrer la bibliothèque dans leurs projets.

Ce rapport présentera donc les différentes étapes de développement du démonstrateur, en fournissant des explications détaillées sur la conception et l'implémentation des modules développés, ainsi que sur la plateforme web regroupant l'ensemble des modules. Des tests et des évaluations seront également effectués pour mesurer la performance et l'efficacité de la solution développée.