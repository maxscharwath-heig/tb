\section{Contexte et introduction}

Le développement de solutions pour la synchronisation et la réplication de données distribuées est un défi majeur dans le domaine de l'informatique moderne. S'attaquant à ce défi, Condensation, une startup innovante, a créé une gamme de logiciels dédiés à garantir la cohérence des données à travers un réseau de n\oe uds. Leur système de synchronisation sans conflit assure l'intégrité des données et une sécurité de bout en bout.

L'architecture initiale de Condensation a été conçue par \textbf{Thomas Lochmatter} et est actuellement maintenue et développée par la startup elle-même. L'enjeu de la synchronisation des données distribuées rend ce projet particulièrement stimulant, notamment en raison des limites des solutions existantes.

Initialement, ce travail de Bachelor visait à collaborer avec Condensation pour développer un démonstrateur utilisant leur système. Cet outil aurait été conçu pour mettre en valeur les fonctionnalités du système et fournir aux développeurs des exemples concrets pour faciliter l'intégration de la bibliothèque dans leurs projets. Cependant, en raison d'une restructuration interne à Condensation, notamment leur décision de réécrire leur système en Rust, cette collaboration n'a pas pu se concrétiser comme prévu.

Face à cette situation, la direction de ce projet de Bachelor a dû être révisée. Au lieu d'utiliser le système existant de Condensation, il a fallu développer de A à Z un système de synchronisation et de réplication de données distribuées. Ce changement d'orientation a entraîné une augmentation significative du travail nécessaire pour ce projet. Cependant, cela a aussi créé une opportunité unique d'apprentissage et de développement de compétences techniques avancées.

En plus du développement de ce nouveau système, une application de création de whiteboard a été conçue pour illustrer son utilisation pratique. Le défi de créer à la fois le système et une application utilisant ce système a été une expérience très stimulante. Je suis particulièrement fier de ces réalisations et je suis enthousiaste à l'idée de les présenter dans ce rapport.

Bien que la collaboration directe avec Condensation n'ait pas eu lieu comme prévu, les principes et l'approche de Condensation ont eu une influence significative sur ce travail. Ce rapport décrira donc les différentes étapes de ce projet réorienté, les choix techniques réalisés, ainsi que les défis rencontrés et les solutions apportées.