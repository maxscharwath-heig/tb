\section{Monorepo}
Un monorepo, ou dépôt mono-référentiel, est une approche de gestion de code source qui consiste à stocker l'ensemble du code et des ressources d'un projet, y compris ses bibliothèques, ses composants et ses applications, dans un seul et même dépôt de contrôle de version. Cette approche contraste avec les polyrepos, où chaque projet, bibliothèque ou composant est stocké dans un dépôt séparé.

\subsection{Avantages}

L'utilisation d'un monorepo présente plusieurs avantages pour les grands projets, notamment :

\begin{itemize}
    \item \textbf{Gestion des dépendances simplifiée :} Dans un monorepo, toutes les versions des dépendances sont centralisées en un seul endroit, ce qui facilite leur gestion et leur mise à jour. Cela permet également de s'assurer que toutes les parties du projet utilisent des versions cohérentes des dépendances, évitant ainsi les problèmes de compatibilité.
    \item \textbf{Collaboration et communication améliorées :} Un monorepo encourage la collaboration entre les développeurs, car ils travaillent tous sur la même base de code. Cela facilite la découverte et la réutilisation des composants et des bibliothèques, et favorise une meilleure compréhension de l'ensemble du projet.
    \item \textbf{Intégration et tests simplifiés :} Dans un monorepo, il est plus facile d'assurer l'intégration et la cohérence des différentes parties du projet, car les modifications apportées à une partie du code peuvent être immédiatement testées et validées dans le contexte global du projet. Cela permet de détecter rapidement les régressions et les problèmes de compatibilité.
    \item \textbf{Historique et traçabilité unifiés :} Un monorepo conserve l'historique de l'ensemble du projet en un seul endroit, ce qui facilite la recherche et la compréhension des modifications apportées au fil du temps. Cela permet également de suivre les dépendances et les interactions entre les différentes parties du projet, ce qui peut être utile pour l'analyse d'impact et le débogage.
\end{itemize}

\subsection{Inconvénients}

Malgré ses avantages, l'utilisation d'un monorepo peut également présenter quelques inconvénients, tels que :

\begin{itemize}
    \item \textbf{Taille du dépôt :} Un monorepo peut devenir très volumineux, ce qui peut ralentir certaines opérations de contrôle de version et compliquer la gestion du dépôt.
    \item \textbf{Complexité de la configuration :} La configuration d'un monorepo peut être plus complexe que celle d'un polyrepo, notamment en ce qui concerne la gestion des dépendances, la configuration des outils de développement et l'automatisation des tests et des déploiements.
    \item \textbf{Séparation des responsabilités :} Dans un monorepo, il peut être plus difficile de maintenir une séparation claire des responsabilités entre les différentes parties du projet, et de contrôler l'accès et les autorisations des développeurs.
\end{itemize}

\section{TurboRepo}

TurboRepo est un système de construction hautes performances pour les projets TypeScript et JavaScript, conçu pour optimiser et accélérer les processus de construction dans les monorepos. Il offre plusieurs fonctionnalités puissantes, telles que :

\begin{itemize}
    \item Construction incrémentale rapide
    \item Mise en cache des calculs locaux
    \item Mise en cache des calculs distribués
    \item Orchestration des tâches locales
    \item Visualisation du graphe des dépendances
    \item Partage du code source
\end{itemize}

L'un des principaux avantages de TurboRepo est sa capacité à minimiser les temps d'inactivité du CPU et à accélérer les tâches grâce à une planification intelligente. Il est conçu pour être adopté de manière incrémentale, ce qui signifie qu'il peut être ajouté à un codebase existant en quelques minutes seulement.

\section{Gestionnaires de paquets et pnpm}

Un gestionnaire de paquets est un outil qui permet d'automatiser l'installation, la mise à niveau, la configuration et la désinstallation de paquets logiciels. Dans le contexte des projets Node.js, un gestionnaire de paquets gère les dépendances et les bibliothèques requises par le projet. Les gestionnaires de paquets aident à simplifier et à organiser le développement en fournissant un moyen structuré et centralisé de gérer les dépendances, ce qui améliore la maintenabilité et la reproductibilité des projets.

Les gestionnaires de paquets les plus couramment utilisés dans l'écosystème JavaScript et Node.js sont npm (Node Package Manager) et yarn. Ces outils permettent aux développeurs d'installer et de gérer les bibliothèques et les modules dont leurs projets ont besoin, en téléchargeant les paquets à partir d'un registre en ligne, généralement le registre npm.

\subsection{pnpm}

pnpm est un gestionnaire de paquets alternatif pour Node.js qui est compatible avec le registre npm. Il se distingue par son approche de stockage des paquets, qui utilise des liens durs et des liens symboliques pour créer un stockage partagé des paquets. Cela permet d'économiser de l'espace disque et de réduire les temps d'installation des paquets.

\subsection{Pourquoi choisir pnpm ?}

Il existe plusieurs raisons pour lesquelles vous pourriez choisir pnpm plutôt que d'autres gestionnaires de paquets comme npm ou yarn :

\begin{itemize}
    \item \textbf{Efficacité du stockage :} Comme mentionné précédemment, pnpm utilise des liens durs et des liens symboliques pour créer un stockage partagé des paquets. Cela signifie que chaque version d'un paquet n'est stockée qu'une seule fois sur le disque, même si elle est utilisée par plusieurs projets. Cela permet de réduire considérablement l'espace disque utilisé par les dépendances du projet.
    \item \textbf{Performances d'installation :} Grâce à son approche de stockage des paquets, pnpm peut réduire les temps d'installation en évitant de télécharger et d'extraire les mêmes paquets plusieurs fois. Les paquets sont simplement liés au projet, ce qui est beaucoup plus rapide que de les copier.
    \item \textbf{Garantie d'isolation :} pnpm garantit que les dépendances d'un projet sont strictement isolées des autres projets, ce qui évite les problèmes causés par les dépendances partagées ou les dépendances non déclarées. Cela permet de détecter et de corriger les erreurs de configuration des dépendances plus tôt dans le processus de développement.
    \item \textbf{Compatibilité :} pnpm est compatible avec le registre npm et les fichiers package.json standard, ce qui signifie qu'il peut être utilisé comme un remplacement direct pour npm ou yarn sans nécessiter de modifications importantes dans la configuration du projet.
\end{itemize}

En somme, pnpm est une alternative intéressante aux gestionnaires de paquets traditionnels comme npm ou yarn. Il offre des avantages en termes d'efficacité du stockage, de performances d'installation et d'isolation des dépendances, tout en restant compatible avec l'écosystème npm existant. L'utilisation de pnpm dans un grand projet peut entraîner des gains significatifs en matière d'espace disque et de temps d'installation, ainsi qu'une meilleure gestion et détection des problèmes liés aux dépendances.

\subsection{Intégration avec TurboRepo}

Comme mentionné précédemment, pnpm peut être utilisé en conjonction avec TurboRepo pour améliorer encore les performances et la gestion des dépendances dans un projet monorepo. TurboRepo se concentre sur l'exécution efficace des tâches de build et de test, tandis que pnpm gère l'installation et la gestion des paquets.

L'intégration de pnpm avec TurboRepo permet de tirer parti des avantages de chacun de ces outils pour créer une expérience de développement plus fluide et performante. TurboRepo peut utiliser les caches de build générés par pnpm pour éviter les tâches de build inutiles et accélérer les pipelines de build. En même temps, pnpm peut tirer parti de la structure monorepo et de la gestion des dépendances centralisée offerte par TurboRepo pour optimiser l'installation et la gestion des paquets.

En conclusion, l'utilisation de pnpm en combinaison avec TurboRepo dans un grand projet peut offrir une solution robuste et performante pour la gestion des dépendances et l'exécution des tâches de build. Les avantages de ces outils peuvent contribuer à améliorer la productivité et la maintenabilité du projet, tout en offrant une expérience de développement plus fluide et plus rapide.


\section{TypeScript}
TypeScript est un langage de programmation développé par Microsoft qui est un sur-ensemble typé de JavaScript. Cela signifie que tout code JavaScript est également du code TypeScript valide. TypeScript ajoute un système de typage statique optionnel à JavaScript, ce qui permet de détecter et d'éviter les erreurs de typage à la compilation plutôt qu'à l'exécution. Le code TypeScript est transcompilé en JavaScript, ce qui le rend compatible avec tous les navigateurs et environnements qui supportent JavaScript.

Dans cette section, nous expliquerons en profondeur ce qu'est TypeScript et pourquoi il est particulièrement utile et nécessaire dans les grands projets.

\subsection{Typage statique}
Le typage statique est l'un des principaux avantages de TypeScript par rapport à JavaScript. Contrairement à JavaScript, qui est un langage à typage dynamique, TypeScript permet de définir des types pour les variables, les fonctions, les objets et les classes. Cela permet au compilateur TypeScript de vérifier la conformité des types lors de la compilation et d'alerter le développeur en cas d'erreurs de typage.

Le typage statique présente plusieurs avantages pour les grands projets :

\begin{itemize}
    \item \textbf{Détection précoce des erreurs :} Le compilateur TypeScript peut détecter les erreurs de typage avant l'exécution du code, ce qui permet de les corriger plus rapidement et de réduire le nombre de bugs en production.
    \item \textbf{Documentation implicite :} Les types fournissent une documentation implicite pour le code, rendant plus facile la compréhension du code par les autres développeurs et facilitant la maintenance et l'évolution du projet.
    \item \textbf{Amélioration de l'autocomplétion et du refactoring :} Les outils de développement modernes peuvent utiliser les informations de type pour fournir une meilleure autocomplétion et faciliter le refactoring du code.
\end{itemize}

\subsection{Compatibilité avec JavaScript}

Comme TypeScript est un sur-ensemble de JavaScript, il est compatible avec tout le code JavaScript existant. Cela signifie que les développeurs peuvent migrer progressivement leur codebase vers TypeScript en ajoutant des types aux parties les plus critiques du code, sans avoir à réécrire l'ensemble de l'application.

Cette compatibilité permet également d'utiliser toutes les bibliothèques et frameworks JavaScript existants, tels que React, Angular et Vue.js, ainsi que les packages npm.

\subsection{Support des fonctionnalités modernes de JavaScript}

TypeScript supporte toutes les fonctionnalités modernes de JavaScript, telles que les classes, les modules, les décorateurs, les fonctions fléchées et les promesses. Il ajoute également des fonctionnalités supplémentaires, telles que les interfaces, les types génériques et les espaces de noms.

Cela permet aux développeurs d'utiliser les dernières fonctionnalités de JavaScript tout en bénéficiant des avantages du typage statique.

\subsection{Scalabilité et maintenabilité}

Dans les grands projets, la complexité du code et la taille de l'équipe de développement peuvent rendre difficile la maintenance et l'évolution du code. TypeScript facilite la gestion de cette complexité en fournissant un système de typage statique et des fonctionnalités avancées de modularisation.

Les avantages de TypeScript en termes de scalabilité et de maintenabilité incluent :

\begin{itemize}
    \item \textbf{Modularité :} TypeScript supporte les modules et les espaces de noms, ce qui permet de diviser le code en unités logiques et de faciliter la réutilisation des composants. Les modules TypeScript peuvent également être intégrés avec les systèmes de modules existants, tels que CommonJS et ES6.
    \item \textbf{Interfaces :} Les interfaces TypeScript permettent de définir des contrats entre les composants du code, garantissant que ces composants respectent les attentes en termes de types et de comportement. Les interfaces sont particulièrement utiles pour définir des API claires et cohérentes entre les différentes parties du code.
    \item \textbf{Types génériques :} Les types génériques permettent de créer des composants réutilisables qui peuvent fonctionner avec différents types de données, sans sacrifier la sécurité du typage. Les types génériques sont couramment utilisés pour définir des structures de données, des fonctions et des classes qui sont indépendantes du type de données qu'elles manipulent.
    \item \textbf{Refactoring :} Les outils de développement modernes peuvent tirer parti des informations de type fournies par TypeScript pour faciliter le refactoring du code. Cela permet de modifier la structure du code sans affecter son comportement, ce qui est essentiel pour la maintenabilité à long terme des grands projets.
\end{itemize}

\subsection{Communauté et écosystème}

TypeScript bénéficie d'une communauté active et en pleine croissance, qui contribue à son développement et à sa popularité. De nombreuses bibliothèques et frameworks JavaScript populaires, tels que React, Angular et Vue.js, supportent TypeScript nativement ou via des fichiers de déclaration de type, ce qui facilite son adoption.

De plus, la majorité des packages npm fournissent des fichiers de déclaration de type pour TypeScript, soit directement dans le package, soit via le dépôt DefinitelyTyped. Cela permet aux développeurs de bénéficier du typage statique lorsqu'ils utilisent des bibliothèques et des frameworks tiers.

\section{Frameworks frontend réactifs}

\subsection{Qu'est-ce qu'un framework frontend réactif ?}

Un framework frontend réactif est une bibliothèque ou un ensemble d'outils permettant de créer des applications Web modernes avec une interface utilisateur dynamique et réactive. Ces frameworks facilitent la manipulation du DOM (Document Object Model), la gestion des états, la communication avec les API backend et l'intégration d'autres fonctionnalités avancées. En outre, ils favorisent la modularité et la réutilisabilité du code en permettant aux développeurs de créer des composants réutilisables et indépendants.

\subsection{Pourquoi les frameworks frontend réactifs sont-ils utiles ?}

Les frameworks frontend réactifs offrent plusieurs avantages pour le développement d'applications Web, notamment :

\paragraph{Réactivité :} Ils permettent de créer des interfaces utilisateur dynamiques et réactives qui s'adaptent automatiquement aux modifications de l'état de l'application ou aux interactions de l'utilisateur.
\paragraph{Modularité et réutilisabilité :} Les frameworks réactifs encouragent la création de composants modulaires et réutilisables, ce qui facilite la maintenance et l'évolution du code, ainsi que la collaboration entre les développeurs.
\paragraph{Productivité :} Ils fournissent des outils et des abstractions pour simplifier le développement et réduire la quantité de code à écrire, ce qui améliore la productivité des développeurs.
\paragraph{Performances :} Les frameworks réactifs optimisent les performances en minimisant les mises à jour du DOM et en utilisant des techniques avancées, telles que la détection de changements, le rendu différé et la mise en cache intelligente.

\subsection{Pourquoi choisir React par rapport aux alternatives ?}

Plusieurs frameworks frontend réactifs sont disponibles, tels que React, Angular et Vue.js. Nous avons choisi React pour ce projet en raison de sa popularité, de sa large compatibilité avec les bibliothèques existantes et de son approche flexible et modulaire. Voici quelques raisons qui ont motivé notre choix :

\paragraph{Popularité :} React est l'un des frameworks frontend les plus populaires et largement utilisés, avec une grande communauté de développeurs et une vaste documentation. Cette popularité signifie que de nombreux problèmes et défis courants ont déjà été résolus par d'autres développeurs, et il existe de nombreuses ressources pour apprendre et se familiariser avec le framework.
\paragraph{Compatibilité avec les bibliothèques :} Grâce à sa popularité, React est compatible avec un grand nombre de bibliothèques et de plugins existants, ce qui facilite l'intégration de fonctionnalités supplémentaires dans notre projet sans avoir à réinventer la roue.
\paragraph{Approche flexible et modulaire :} React se concentre sur la création de composants réutilisables et indépendants, ce qui facilite la modularité et la réutilisabilité du code. Cette approche flexible permet de créer des applications à l'architecture claire et maintenable, tout en facilitant la collaboration entre les développeurs.
\paragraph{Performances :} React utilise un algorithme de réconciliation efficace et un DOM virtuel pour minimiser les mises à jour du DOM réel, ce qui améliore les performances de l'application. De plus, React prend en charge des fonctionnalités telles que le découpage de code et le chargement différé pour optimiser davantage les performances.

\paragraph{Ecosystème :} L'écosystème de React est riche et diversifié, offrant un grand nombre de bibliothèques, d'outils et de solutions pour résoudre divers problèmes de développement. Cela permet aux développeurs de se concentrer sur la logique métier de l'application plutôt que sur les détails techniques.

\paragraph{Mise à jour et maintenance :} React est développé et maintenu par Facebook, qui l'utilise dans ses propres applications. Cela signifie que React bénéficie d'un soutien solide, de mises à jour régulières et d'une maintenance à long terme.

\section{Utilisation de Tailwind CSS}

Tailwind CSS est un framework CSS moderne et très populaire qui se concentre sur l'utilité, la flexibilité et la personnalisation. Il est conçu pour faciliter la création de designs responsifs et cohérents en fournissant un ensemble de classes CSS préconfigurées qui peuvent être facilement combinées pour construire rapidement des interfaces utilisateur. Dans cette section, nous discuterons des avantages et des inconvénients de l'utilisation de Tailwind CSS, ainsi que des raisons pour lesquelles il a été choisi pour ce projet.

\subsection{Avantages}

\paragraph{Utilitaires basés sur des classes}

Tailwind CSS utilise une approche basée sur des classes utilitaires, ce qui permet de créer des styles directement dans le code HTML. Cela facilite la compréhension du style appliqué à un élément, car il n'est pas nécessaire de naviguer entre plusieurs fichiers CSS. Cette approche permet également une meilleure réutilisabilité et une maintenance plus facile des styles, car les classes utilitaires sont cohérentes et préconfigurées.

\paragraph{Personnalisation}

Tailwind CSS offre un haut degré de personnalisation grâce à son fichier de configuration. Les développeurs peuvent définir leurs propres valeurs pour les couleurs, les polices, les espacements et d'autres propriétés CSS. Cette personnalisation permet de créer des designs uniques et cohérents tout en bénéficiant de la facilité d'utilisation et de la rapidité de développement offerte par les classes utilitaires de Tailwind.

\paragraph{Optimisation des performances}

Tailwind CSS est livré avec des outils d'optimisation intégrés qui aident à réduire la taille du fichier CSS final. Par exemple, il utilise PurgeCSS pour supprimer automatiquement les classes CSS inutilisées du fichier de production. Cela garantit que seules les classes réellement utilisées dans le projet sont incluses dans le fichier CSS, réduisant ainsi la taille du fichier et améliorant les performances.

\subsection{Inconvénients}

\paragraph{Courbe d'apprentissage}

La documentation de Tailwind CSS est très complète et fournit de nombreuses ressources pour apprendre à utiliser le framework. S'il on est familiarisé avec Bootstrap, il est relativement facile de se familiariser avec Tailwind CSS. Cependant, il peut être difficile pour les développeurs qui n'ont jamais utilisé de framework CSS de comprendre comment fonctionne Tailwind CSS et comment l'utiliser pour créer des designs responsifs et cohérents. Il est donc judicieux de passer du temps à lire la documentation et à explorer les exemples fournis avant de commencer à utiliser Tailwind CSS dans un projet.

\paragraph{Verbosité}

L'utilisation de classes utilitaires peut rendre le code HTML plus verbeux, car chaque élément doit inclure toutes les classes nécessaires pour appliquer le style souhaité. Cela peut rendre le code HTML plus difficile à lire et à maintenir. Cependant, ce problème peut être atténué en extrayant les styles réutilisables dans des composants ou en utilisant des directives CSS personnalisées.

\subsection{Pourquoi choisir Tailwind CSS pour ce projet?}

Le choix de Tailwind CSS pour ce projet s'est avéré être une décision judicieuse, car il offre un grand nombre d'avantages et de fonctionnalités qui répondent aux besoins du projet. En plus de ses avantages évidents, tels que la facilité d'utilisation et la rapidité de développement, Tailwind CSS offre également une grande flexibilité et une personnalisation poussée, ce qui permet de créer des designs uniques et cohérents. De plus, Tailwind CSS est livré avec des outils d'optimisation intégrés qui aident à réduire la taille du fichier CSS final, ce qui améliore les performances de l'application.

\section{Gestion des états avec les stores React et l'utilisation de Zustand}

Dans les applications React, la gestion de l'état est un aspect essentiel qui permet de maintenir et de contrôler les données à travers les composants de l'application. React offre plusieurs méthodes pour gérer l'état, notamment les hooks d'état locaux, le contexte et les providers. Cependant, ces solutions natives peuvent devenir rapidement complexes et difficiles à maintenir lorsqu'on travaille sur de grandes applications ou avec des états globaux. C'est là qu'interviennent les bibliothèques de gestion d'état tierces, telles que Redux, MobX et Zustand.

\subsection{Zustand: une alternative aux React Providers}

Zustand est une bibliothèque de gestion d'état minimaliste et légère pour les applications React. Elle se distingue par sa simplicité, sa flexibilité et sa facilité d'utilisation. Contrairement aux React Providers, qui reposent sur le mécanisme de contexte de React pour partager et consommer l'état, Zustand utilise un système de stores pour gérer et distribuer l'état de manière centralisée.

Les stores Zustand sont des objets JavaScript qui contiennent l'état et les fonctions associées pour le manipuler. Les composants React peuvent s'abonner à ces stores et accéder à l'état et aux fonctions sans avoir besoin de passer par des providers ou des consommateurs. Cette approche simplifie la gestion de l'état en éliminant la nécessité de gérer le contexte ou de propager les données à travers les composants de manière manuelle.

\subsection{Avantages de l'utilisation de Zustand}

Voici quelques-uns des avantages de l'utilisation de Zustand pour la gestion de l'état dans les applications React:

\paragraph{Simplicité :} Zustand offre une API simple et concise pour définir, manipuler et consommer l'état. Il n'y a pas besoin de configurer des providers, des reducers ou des actions comme avec d'autres solutions de gestion d'état. Les développeurs peuvent ainsi se concentrer sur la logique métier et l'expérience utilisateur.

\paragraph{Flexibilité :} Les stores Zustand peuvent être créés et combinés de manière modulaire pour répondre aux besoins spécifiques de l'application. Cette flexibilité permet d'organiser et de structurer l'état de manière logique et maintenable.

\paragraph{Performance :} Zustand utilise un mécanisme d'abonnement pour mettre à jour sélectivement les composants qui consomment l'état. Cela garantit que seuls les composants concernés par les modifications d'état sont re-rendus, ce qui améliore les performances de l'application.

\paragraph{{Interopérabilité :} Zustand est compatible avec les hooks React et peut être facilement intégré à d'autres bibliothèques.