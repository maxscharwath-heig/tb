Dans un contexte où le travail à distance et la collaboration en ligne sont de plus en plus courants, la question de la souveraineté des données et de la sécurité est cruciale. Les systèmes centralisés, souvent sous le contrôle des géants du Web, ont des implications en termes de dépendance et de protection des données. En réponse à cette problématique, ce travail de Bachelor introduit deux réalisations : \textbf{DDnet}, un système de synchronisation de données décentralisées, et \textbf{Describble}, une application de tableau blanc en ligne conçue pour illustrer les capacités de ce système.

\textbf{DDnet} est un système de synchronisation de données décentralisées. Il est basé sur une approche Local-First, priorisant l'accès et la modification des données locales. Pour une synchronisation des données sécurisée et sans conflit, DDnet intègre les CRDT (Conflict-free Replicated Data Type). Il fonctionne sur un réseau décentralisé peer-to-peer, renforcé par des techniques de cryptographie et de chiffrement de bout en bout. DDnet est conçu pour être flexible et peut travailler avec tout type de données structurées, comme le JSON. Ce système est open source et est disponible sur \url{https://www.npmjs.com/package/@describble/ddnet}.

\asterism

Sur cette base, \textbf{Describble} est conçu comme une application de tableau blanc en ligne pour montrer les capacités de DDnet. Describble offre un environnement collaboratif en temps réel pour le brainstorming et la réflexion. En utilisant un espace de dessin illimité, il permet aux utilisateurs de partager et de développer des idées de manière interactive. Les outils de dessin et un système de calques offrent aux utilisateurs la possibilité de structurer leurs idées de manière créative. Un système de gestion des droits d'accès est mis en place pour assurer la sécurité des informations partagées. Describble est accessible sur \url{https://describble.io}.

\asterism

Ensemble, DDnet et Describble représentent une exploration concrète des possibilités offertes par la décentralisation et la collaboration en temps réel. Ils montrent qu'il est possible de créer des outils puissants et indépendants qui respectent la souveraineté des données des utilisateurs.