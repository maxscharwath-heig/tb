Ce travail de Bachelor présente deux réalisations : 

\textbf{Describble}, une application de tableau blanc en ligne, et \textbf{DDnet}, un système de synchronisation de données décentralisées.

\textbf{Describble} offre un environnement collaboratif pour la réflexion et le brainstorming. Utilisant un espace de dessin illimité, les participants peuvent partager et développer des idées de manière interactive. Les dessins sont créés à l'aide de vecteurs, ce qui permet une flexibilité des modifications. Un système de calques offre une organisation supplémentaire, permettant de séparer et de regrouper les éléments du dessin. Describble fournit également une variété d'outils de dessin pour faciliter la création. La sécurité est assurée par un système de gestion des droits d'accès, offrant un contrôle sur qui peut voir et modifier le contenu. Describble est accessible sur \url{https://describble.io}.

\asterism

Le deuxième projet, \textbf{DDnet}, est un système de synchronisation de données décentralisées développé en parallèle. DDnet utilise une approche "local-first", privilégiant l'accès et la modification des données locales. DDnet intègre les CRDT (Conflict-free Replicated Data Type) pour assurer une synchronisation des données sans conflit, en temps réel. Cela permet à plusieurs utilisateurs de modifier les données simultanément sans conflits. En outre, DDnet utilise un réseau décentralisé peer-to-peer, renforcé par des techniques de cryptographie et de chiffrement de bout en bout pour assurer la sécurité des données. DDnet est conçu pour être flexible et peut travailler avec tout type de données structurées, comme le JSON. DDnet est open source et est disponible sur \url{https://www.npmjs.com/package/@describble/ddnet}.

\asterism

Ensemble, ces deux projets illustrent une vision pour un avenir du travail collaboratif en ligne qui n'est pas dépendant des systèmes centralisés traditionnels gérés par les géants du Web. En utilisant une technologie décentralisée et des principes de 'local-first', DDnet et Describble permettent une véritable collaboration en temps réel tout en garantissant la souveraineté des données. Ce travail démontre qu'il est possible de créer des outils puissants qui respectent l'autonomie des utilisateurs et leur permettent de contrôler leurs propres données.