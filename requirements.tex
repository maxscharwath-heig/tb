\section*{Présentation}
L'objectif principal de ce travail est de contribuer au développement de la start-up Condensation en construisant un démonstrateur qui permettra de mettre en lumière les fonctionnalités du système, et de fournir aux développeurs des exemples concrets pour les aider à implémenter efficacement la librairie Condensation.

La start-up Condensation est spécialisée dans le développement d'une solution pour la synchronisation de données distribuées. La bibliothèque de Condensation offre une approche innovante pour garantir la cohérence des données en utilisant un système de synchronisation conflict-free qui garantit l'intégrité des données et la sécurité de bout en bout.

Pour atteindre cet objectif, une application sera développée pour exploiter les capacités du système de Condensation.

Premièrement, un Whiteboard sera développé pour illustrer les fonctionnalités du système.

Dans un second temps, des outils de développement seront développés pour visualiser et éditer les données stockées dans le système de Condensation.

Pour finir, une documentation détaillée sera fournie pour expliquer comment utiliser la librairie Condensation dans ses projets.
\newpage
\section*{Livrables}
Les livrables à réaliser pour ce travail de bachelor sont les suivants:

\begin{itemize}
    \item Un cahier des charges;
    \item Un rapport intermédiaire;
    \item Un rapport final;
    \item Un résumé publiable;
    \item Un poster;
    \item Un démonstrateur fonctionnel basé sur la librairie de Condensation permettant de mettre en évidence les fonctionnalités du système;
    \item Des outils de développement permettant de visualiser et éditer les données stockées dans le système de Condensation;
    \item Une documentation détaillée pour l'utilisation de la librairie Condensation.
\end{itemize}

\section*{Objectifs}
Ce projet va répondre à plusieurs objectifs, répartis en 3 catégories : les objectifs \textbf{fonctionnels}, qui décrivent les fonctionnalités attendues du système, les objectifs \textbf{non fonctionnels}, qui décrivent les caractéristiques de performance, de sécurité et d'accessibilité du système, et les objectifs \textbf{complémentaires}, qui décrivent les fonctionnalités souhaitables mais non-essentielles du système.

\subsection*{Objectifs \guillemotleft fonctionnels\guillemotright}

\begin{itemize}
    \item Développer un Whiteboard utilisant le système de Condensation pour illustrer clairement ses fonctionnalités. Le Whiteboard devra suivre les spécifications définies dans la section \sectionref{sec:user-stories-whiteboard}.
    \item Fournir des outils permettant de visualiser et éditer les données stockées dans le système de Condensation. Les outils de développement devront suivre les spécifications définies dans la section \sectionref{sec:user-stories-devtools}.
    \item Fournir une documentation précise pour l'utilisation de la librairie Condensation avec des exemples afin d'aider les développeurs à l'intégrer dans leurs projets.
\end{itemize}

\subsection*{Objectifs \guillemotleft non fonctionnels\guillemotright}

\begin{itemize}
    \item Concevoir une interface utilisateur intuitive, afin de permettre aux développeurs de tout de suite comprendre le fonctionnement des exemples.
    \item Garantir que le système peut être facilement étendu et adapté aux besoins futurs en utilisant des architectures évolutives et modulaires.
    \item Assurer que le code source soit bien structuré, propre et documenté, afin de faciliter la lecture et la compréhension par d'autres développeurs.
\end{itemize}

\subsection*{Objectifs \guillemotleft complémentaires\guillemotright}

\begin{itemize}
    \item Ajouter des fonctionnalités en temps réel tel que la visualisation des curseurs des utilisateurs.
\end{itemize}

\section*{User stories}
Les user stories sont des descriptions de fonctionnalités du système, écrites sous la forme d'une phrase simple, qui permettent de décrire les besoins des utilisateurs. Elles sont utilisées pour définir les fonctionnalités du système, et sont donc un outil précieux pour la planification et l'exécution du projet.

\subsection*{Whiteboard \label{sec:user-stories-whiteboard}}

\begin{itemize}
    \item En tant qu'utilisateur, je veux pouvoir éditer un Whiteboard afin de pouvoir créer et modifier du contenu.
    \item En tant qu'utilisateur, je veux pouvoir partager un Whiteboard avec un lien unique pour faciliter la collaboration avec d'autres personnes.
    \item En tant qu'utilisateur, je veux pouvoir ajouter et supprimer des éléments sur le Whiteboard afin de gérer le contenu présenté.
    \item En tant qu'utilisateur, je veux pouvoir déplacer des éléments sur le Whiteboard pour organiser l'espace selon mes besoins.
    \item En tant qu'utilisateur, je veux pouvoir éditer le Whiteboard en mode “hors-ligne” pour pouvoir travailler sans une connexion internet stable.
    \item En tant qu'utilisateur, je veux pouvoir éditer le Whiteboard simultanément avec d'autres utilisateurs pour faciliter la collaboration en temps réel.
    \item En tant qu'utilisateur, je veux pouvoir choisir où stocker mes données pour assurer la confidentialité et la sécurité de mes informations.
    \item En tant qu'utilisateur, je veux pouvoir voir l'historique des modifications apportées au Whiteboard pour comprendre l'évolution du contenu et peut-être revenir à une version précédente.
\end{itemize}

\subsection*{Outil de développement \label{sec:user-stories-devtools}}

L'outil de développement est une application qui permettra de visualiser et d'éditer les données stockées dans le système. Il permettra aux développeurs de comprendre comment les données sont stockées.

\begin{itemize}
    \item En tant que développeur, je veux pouvoir visualiser les données stockées sous forme d'arbre de hash pour comprendre la structure des données.
    \item En tant que développeur, je veux pouvoir visualiser les données stockées sous forme de document pour comprendre le contenu.
    \item En tant que développeur, je veux pouvoir éditer les données stockées sous forme de document pour modifier le contenu si nécessaire.
    \item En tant que développeur, je veux pouvoir utiliser l'outil de développement sous forme d'extension pour mon navigateur Chromium pour faciliter l'accès et l'utilisation de l'outil.
\end{itemize}
\newpage
\section*{Contraintes et risques}
Le projet est soumis à certaines contraintes et risques qu'il est important de prendre en compte pour garantir sa réussite.
La livraison de la librairie de Condensation est un élément clé du projet et sa disponibilité en temps voulu est essentielle pour avancer dans le développement des fonctionnalités prévues. 
Bien que Condensation ait promis de livrer la librairie en plusieurs étapes, il est possible que des retards surviennent, ce qui pourrait avoir un impact sur la planification du projet.
Pour minimiser ces risques, il a été décidé de planifier la livraison de la librairie en étapes successives pour pouvoir implémenter les fonctionnalités du projet au fur et à mesure de la disponibilité de la librairie.
Dans le cas où la livraison de la librairie serait retardée, et mettrait en péril la réussite du projet, une date limite a été fixée pour utiliser une alternative à la librairie de Condensation.

\section*{Planification}
La planification du projet est divisée en plusieurs étapes principales. Ces étapes servent de jalons pour évaluer la progression du projet et permettent d'adapter la planification en fonction des besoins et des imprévus qui pourraient survenir au cours du développement.

\begin{itemize}
    \item Analyse des besoins et rédaction du cahier des charges.
    \item Recherche sur l'état de l'art des technologies de synchronisation de données distribuées et évaluation des alternatives possibles à la librairie Condensation.
    \item Formation sur la technologie Condensation et les mécanismes de synchronisation.
    \item Développement du Whiteboard en fonction des spécifications définies dans les user stories \sectionref{sec:user-stories-whiteboard}.
    \item Développement des outils de développement en fonction des spécifications définies dans les user stories \sectionref{sec:user-stories-devtools}.
    \item Test et validation des fonctionnalités développées.
    \item Rédaction de la documentation pour l'utilisation de la librairie Condensation.
    \item Préparation et présentation du rapport final, du résumé publiable et du poster.
    \item Date limite pour décider d'un plan B, si la livraison de la librairie Condensation est retardée ou compromise.
    \item Si nécessaire, mise en \oe{}uvre d'un plan B, basé sur les alternatives identifiées lors de la recherche sur l'état de l'art.
\end{itemize}

\section*{Planification de livraison de la librairie Condensation}
La livraison de la librairie de Condensation étant en attente, le projet sera implémenté en plusieurs étapes en fonction des livraisons successives de la librairie.
\begin{itemize}
    \item \textbf{Étape 0 (22 mars 2023)} : Livraison de l'interface de l'API de la libraire de Condensation. Cette étape permettra de développer les fonctionnalités en simulant la librairie de Condensation.
    \item \textbf{Étape 1 (17 avril 2023)} : Gérer les données en local et visualiser les données dans la forme de document et dans la forme de hash. Cette étape permettra de visualiser les données dans un document et de les stocker sous forme de hash en local.
    \item \textbf{Étape 2 (15 mai 2023)} : Gérer les données en remote et visualiser les hash / les messages stockés sur le serveur. Cette étape permettra de stocker les données sur un serveur distant et de les visualiser sous forme de hash / messages.
    \item \textbf{Étape 3 (12 juin 2023)} : Créer un second acteur qui peut communiquer avec un flux d'information sécurisé. Cette étape permettra la communication sécurisée entre deux acteurs en utilisant la librairie de Condensation.
    \item \textbf{Étape 4 (10 juillet 2023)} : Rendre l'expérience collaborative en live avec la résolution de conflits. Cette étape permettra la collaboration en temps réel entre plusieurs acteurs et la résolution de conflits éventuels.
\end{itemize}

\section*{Date limite pour un plan B}
Tenant compte des différentes étapes de la planification et de la nécessité de
prévoir suffisamment de temps pour la mise en \oe{}uvre d'un plan B, une date limite
pour décider d'un plan B est fixée au \textbf{19 juin 2023}. Cette date offre un
délai approprié pour évaluer la situation et mettre en place une alternative si
nécessaire, tout en maintenant un calendrier raisonnable pour achever le projet
dans les temps impartis.


\chapter*{Rectification du cahier des charges}
\addcontentsline{toc}{chapter}{Rectification du cahier des charges}

Suite à une discussion approfondie avec mon superviseur, le Professeur Bertil Chapuis, il est devenu évident qu'une rectification du cahier des charges était nécessaire. Cela a été principalement dû à l'absence de la bibliothèque Condensation nécessaire pour la réalisation du projet initial. Ainsi, une réorientation importante du projet a été mise en oeuvre.

L'outil de développement et les exemples d'utilisation, qui étaient intimement liés à la bibliothèque de Condensation, n'ont pas pu être réalisés. Par contre, le Whiteboard, qui était prévu initialement comme un simple démonstrateur des capacités du système, a été largement développé et enrichi pour devenir un produit complet en soi.

Des fonctionnalités supplémentaires ont été ajoutées au Whiteboard, notamment un plus grand nombre d'outils de dessin, une gestion de documents, une internationalisation et un déploiement en production. Le système sur lequel se base le Whiteboard a été développé entièrement à partir de zéro, remplaçant ainsi la bibliothèque de Condensation initialement prévue.

Ce système a été conçu dès le départ pour être open source. Par conséquent, un effort supplémentaire a été consenti pour l'écrire de manière à ce qu'il soit public, en le commentant et en le documentant de manière adéquate. Le système a été rendu accessible sur le registre de paquets \gls{npm}, pour faciliter l'utilisation de la librairie par le public.

Ainsi, le cahier des charges a été modifié pour refléter ces changements majeurs :
\section*{Livrables}

Les livrables à réaliser pour ce travail de bachelor sont les suivants :

\begin{itemize}
    \item Un cahier des charges;
    \item un rapport intermédiaire;
    \item un rapport final;
    \item un résumé publiable;
    \item un poster;
    \item un système complet de Whiteboard, avec des outils de dessin avancés, une gestion de documents et une internationalisation;
    \item un déploiement en production du Whiteboard;
    \item un système de synchronisation de données distribuées développé de A à Z, conçu pour être open source et utilisable par le public.
\end{itemize}

\section*{Objectifs}

\subsection*{Objectifs \guillemotleft fonctionnels\guillemotright}

\begin{itemize}
    \item Développer un Whiteboard utilisant le système de synchronisation de données distribuées développé spécifiquement pour ce projet, et le rendre complet avec des outils de dessin avancés, une gestion de documents, une internationalisation, et un déploiement en production.
\end{itemize}

\subsection*{Objectifs \guillemotleft non fonctionnels\guillemotright}

\begin{itemize}
    \item Concevoir une interface utilisateur intuitive, afin de permettre aux utilisateurs de tout de suite comprendre le fonctionnement du Whiteboard.
    \item Garantir que le système peut être facilement étendu et adapté aux besoins futurs en utilisant des architectures évolutives et modulaires.
    \item Assurer que le code source soit bien structuré, propre et documenté, afin de faciliter la lecture et la compréhension par d'autres développeurs.
\end{itemize}

\subsection*{Objectifs \guillemotleft complémentaires\guillemotright}

\begin{itemize}
    \item Ajouter des fonctionnalités en temps réel tel que la visualisation des curseurs des utilisateurs.
\end{itemize}

\section*{User stories}
\subsubsection*{Whiteboard}

\begin{itemize}
    \item En tant qu'utilisateur, je veux pouvoir éditer un Whiteboard afin de pouvoir créer et modifier du contenu.
    \item En tant qu'utilisateur, je veux pouvoir ajouter, modifier, déplacer, et supprimer des éléments sur le Whiteboard pour gérer le contenu présenté.
    \item En tant qu'utilisateur, je veux pouvoir changer le style d'un élément sur le Whiteboard pour améliorer l'esthétique et la clarté de mon contenu.
    \item En tant qu'utilisateur, je veux pouvoir utiliser un système d'annulation/rétablissement (undo/redo) pour corriger facilement mes erreurs.
    \item En tant qu'utilisateur, je veux pouvoir éditer le Whiteboard en mode “hors-ligne” pour pouvoir travailler sans une connexion internet stable.
    \item En tant qu'utilisateur, je veux pouvoir éditer le Whiteboard simultanément avec d'autres utilisateurs pour faciliter la collaboration en temps réel.
    \item En tant qu'utilisateur, je veux pouvoir gérer finement les accès à mon document pour contrôler qui peut voir et modifier mon contenu.
    \item En tant qu'utilisateur, je veux pouvoir lister tous les documents disponibles sur ma machine pour faciliter la gestion et l'accès à mes documents.
    \item En tant qu'utilisateur, je veux pouvoir utiliser l'application avec un système de session pour maintenir mes préférences et mon état de travail entre les utilisations.
    \item En tant qu'utilisateur, je veux pouvoir utiliser l'application avec plusieurs comptes pour gérer différents ensembles de documents et préférences.
\end{itemize}

\subsubsection*{Système de synchronisation de données distribuées}

\begin{itemize}
    \item En tant que développeur, je veux pouvoir utiliser une librairie de synchronisation de données distribuées pour faciliter le développement d'applications nécessitant ce genre de fonctionnalités.
    \item En tant que développeur, je veux que la librairie de synchronisation de données distribuées soit bien documentée et commentée pour faciliter sa compréhension et son utilisation.
    \item En tant que développeur, je veux que la librairie de synchronisation de données distribuées soit accessible sur le registre de paquets \gls{npm} pour en faciliter l'installation et la gestion des dépendances.
    \item En tant que développeur, je veux pouvoir observer le comportement du système de synchronisation de données en temps réel pour comprendre son fonctionnement et déboguer les problèmes.
    \item En tant que développeur, je veux pouvoir tester facilement le système de synchronisation de données distribuées pour garantir sa fiabilité et sa robustesse.
    \item En tant que développeur, je veux pouvoir étendre et personnaliser le système de synchronisation de données distribuées pour l'adapter aux besoins spécifiques de mes projets.
    \item En tant que développeur, je veux pouvoir utiliser le système de synchronisation de données distribuées dans une variété d'environnements (navigateur, \gls{Node.js}, etc.) pour maximiser sa flexibilité et sa portabilité.
\end{itemize}

\section*{Nouvelle Planification du Projet (à partir de mi-juin 2023)}

Afin d'assurer la réalisation des objectifs du projet, une nouvelle
planification a été conçue pour la période allant de mi-juin à fin juillet 2023.

\begin{itemize}
    \item \textbf{Mi-juin 2023 :} Prise de décision de réorienter le projet et définition de la nouvelle planification.
    \item \textbf{Mi-juin à début juillet 2023 (3 semaines) :} Concentration totale sur le développement du système de synchronisation de données distribuées. Cela comprend le prototypage, l'implémentation et les tests préliminaires du système.
    \item \textbf{Début juillet à mi-juillet 2023 (2 semaines) :} Intégration du système de synchronisation de données dans l'application Whiteboard. Continuation du développement de l'application en parallèle avec le système de synchronisation, ajout et amélioration des fonctionnalités existantes du Whiteboard.
    \item \textbf{Mi-juillet à 17 juillet 2023 (1 semaine) :} Finalisation du développement, réalisation des tests finaux, correction de bogues et optimisation du code.
    \item \textbf{17 juillet à 27 juillet 2023 :} Période dédiée à la correction de bogues mineurs, l'optimisation du code, la rédaction du rapport final, du résumé publiable et du poster.
\end{itemize}

Cette planification a été conçue pour maximiser l'efficacité du développement et garantir la réalisation du projet dans le délai imparti.