\section*{Présentation}
L'objectif principal de ce travail est de contribuer au développement de la startup Condensation en construisant un démonstrateur qui permettra de mettre en lumière les fonctionnalités du système, et de fournir aux développeurs des exemples concrets pour les aider à implémenter efficacement la librairie Condensation.

Pour atteindre cet objectif, plusieurs exemples seront développés pour exploiter les capacités du système de Condensation. Ces exemples seront regroupés sur une plateforme web intuitive, accompagnés d'une documentation qui fournira des explications détaillées sur l'utilisation de chaque exemple, ainsi que des tutoriels pour aider les développeurs à intégrer la librairie dans leurs projets.

Le résultat final de ce travail sera donc une solution clé en main pour les développeurs qui souhaitent utiliser le système de Condensation. Le démonstrateur développé constituera un outil précieux pour faciliter l'adoption du système, et contribuera ainsi au développement de la librairie en tant que solution de référence pour la synchronisation de données distribuées.

\section*{Livrables}
Les livrables à réaliser pour ce travail de bachelor sont les suivants:

\begin{itemize}
    \item Un rapport intermédiaire,
    \item Un rapport final,
    \item Un résumé publiable,
    \item Un poster,
    \item Un démonstrateur fonctionnel basé sur la librairie de Condensation permettant de mettre en évidence les fonctionnalités du système.
    \item Une plateforme web permettant de regrouper l'ensemble des modules développés, avec une documentation et tutoriels détaillés.
\end{itemize}

\section*{Objectifs}
Ce projet va répondre à plusieurs objectifs, répartis en 3 catégories : les objectifs \textbf{fonctionnels}, qui décrivent les fonctionnalités attendues du système, les objectifs \textbf{non-fonctionnels}, qui décrivent les caractéristiques de performance, de sécurité et d'accessibilité du système, et les objectifs \textbf{complémentaires}, qui décrivent les fonctionnalités souhaitables mais non-essentielles du système.

\subsection*{Objectifs \guillemotleft fonctionnels\guillemotright}

\begin{itemize}
    \item Développer plusieurs modules utilisant le système de Condensation qui permettront d'illustrer les fonctionnalités du système et de montrer aux développeurs comment les implémenter de manière efficace.
    \item Un exemple conséquent permettant l'édition d'un Whiteboard simultanément avec d'autres navigateurs, avec gestion des déconnexions inattendues, support de l'édition en mode “hors-ligne”, et doit pouvoir être partagé facilement.
    \item Regrouper les exemples sur une plateforme web intuitive, offrant une expérience utilisateur optimale.
    \item Fournir une documentation précise pour chaque module, comprenant des tutoriels pour aider les développeurs à comprendre comment les intégrer dans leurs projets.
    \item Fournir deux serveurs de stockage de données, pour permettre aux utilisateurs de choisir où stocker leurs données.
\end{itemize}


\subsection*{Objectifs \guillemotleft non-fonctionnels\guillemotright}

\begin{itemize}
    \item Concevoir une interface utilisateur intuitive, afin de permettre aux développeurs de tout de suite comprendre le fonctionnement des exemples.
    \item Garantir que le système peut être facilement étendu et adapté aux besoins futurs en utilisant des architectures évolutives et modulaires.
    \item S'assurer que le système est accessible aux utilisateurs de toutes les capacités, notamment en respectant les normes d'accessibilité et en prenant en compte les différents types d'appareils et de plateformes.
\end{itemize}

\subsection*{Objectifs \guillemotleft complémentaires\guillemotright}

\begin{itemize}
    \item Ajouter d'autres exemples plus complexes.
    \item Offrir une interface multilingue pour la plateforme, afin de la rendre accessible à un public plus large.
\end{itemize}

\section*{Contraintes}
Cependant, il est important de noter que la librairie de Condensation est encore en développement. Pour éviter de perdre du temps sur le développement, un système de mock\footnote{Un mock de l'anglais \textit{mock-up}, qui signifie \textit{maquette}. Un mock est un objet qui simule le comportement d'un autre objet dans un contexte donné.} sera introduit pour simuler les interactions avec la librairie, tout en conservant la possibilité de l'intégrer facilement une fois qu'elle sera prête. Cette contrainte sera prise en compte dans la planification et l'exécution du projet.