\documentclass[
    iict, % Saisir le nom de l'institut rattaché
    il, % Saisir le nom de l'orientation
    % confidential, % Décommentez si le travail est confidentiel
]{heig-tb}

\usepackage[nooldvoltagedirection,european,americaninductors]{circuitikz}

\signature{mbernasconi.svg} % Remplacer par votre propre signature vectorielle.

\makenomenclature
\makenoidxglossaries
\makeindex

\addbibresource{bibliography.bib}

\usepackage{etoolbox}
\renewcommand\nomgroup[1]{%
  \item[\bfseries
  \ifstrequal{#1}{A}{Constantes physiques}{%
  \ifstrequal{#1}{B}{Groupes}{%
  \ifstrequal{#1}{C}{Autres Symboles}{}}}%
]}

\newcommand{\nomunit}[1]{%
\renewcommand{\nomentryend}{\hspace*{\fill}#1}}

\nomenclature[A, 02]{\(c\)}{\href{https://physics.nist.gov/cgi-bin/cuu/Value?c}
{Vitesse de la lumière dans le vide}
\nomunit{\SI{299792458}{\meter\per\second}}}

\nomenclature[A, 03]{\(h\)}{\href{https://physics.nist.gov/cgi-bin/cuu/Value?h}
{Constante de Planck}
\nomunit{\SI[group-digits=false]{6.62607015e-34}{\joule\per\hertz}}}

\nomenclature[A, 01]{\(G\)}{\href{https://physics.nist.gov/cgi-bin/cuu/Value?bg}
{Constante de gravitation universelle}
\nomunit{\SI[group-digits=false]{6.67430e-11}{\meter\cubed\per\kilogram\per\second\squared}}}

\nomenclature[B, 03]{\(\mathbb{R}\)}{Nombres réels}
\nomenclature[B, 02]{\(\mathbb{C}\)}{Nombres complexes}
\nomenclature[B, 01]{\(\mathbb{H}\)}{Quaternions}

\nomenclature[C]{\(V\)}{Volume constant}
\nomenclature[C]{\(\rho\)}{Indice de frottement sec}

\newacronym{gcd}{GCD}{Plus grand diviseur commun}
\newacronym{lcm}{LCM}{Plus petit multiple commun}

\newglossaryentry{MitM}{
    name=MitM,
    description={Man in the Middle, attaque qui consiste à se placer entre deux interlocuteurs afin de récupérer des informations}
}

\newglossaryentry{ECDH}{
    name=ECDH,
    description={Elliptic Curve Diffie-Hellman, algorithme de génération de clés utilisant des courbes elliptiques}
}

\newglossaryentry{ECDSA}{
    name=ECDSA,
    description={Elliptic Curve Digital Signature Algorithm, algorithme de signature numérique utilisant des courbes elliptiques}
}

\newglossaryentry{PBKDF2}{
    name=PBKDF2,
    description={Password-Based Key Derivation Function 2, algorithme de dérivation de clés à partir d'un mot de passe}
}

\newglossaryentry{AES}{
    name=AES,
    description={Advanced Encryption Standard, algorithme de chiffrement par blocs}
}

\newglossaryentry{PWA}{
    name=PWA,
    description={Progressive Web Application, application web qui peut être installée sur un appareil et utilisée hors ligne}
}

\newglossaryentry{W3C}{
    name=W3C,
    description={World Wide Web Consortium, organisation à but non lucratif qui développe des standards pour le web}
}

\newglossaryentry{CRDT}{
    name=CRDT,
    description={Conflict-free Replicated Data Type, structure de données répliquée qui peut être mise à jour de manière indépendante et converger vers un état cohérent}
}

\newglossaryentry{DOM}{
    name=DOM,
    description={Document Object Model, interface de programmation pour les documents HTML et XML}
}

\newglossaryentry{TCP}{
    name=TCP,
    description={Transmission Control Protocol, protocole de transport fiable}
}

\newglossaryentry{WebRTC}{
    name=WebRTC,
    description={Web Real-Time Communication, API pour la communication en temps réel entre navigateurs}
}

\newglossaryentry{WebSocket}{
    name=WebSocket,
    description={Protocole de communication bidirectionnel, full-duplex, sur un seul socket TCP}
}

\newglossaryentry{DDnet}{
    name=DDnet,
    description={Decentralized Document Network, système réalisé dans le cadre de ce projet}
}

\newglossaryentry{Describble}{
    name=Describble,
    description={Nom de l'application réalisée dans le cadre de ce projet}
}

\newglossaryentry{DNS}{
    name=DNS,
    description={Domain Name System, système de noms de domaine, permet de faire la correspondance entre un nom de domaine et une adresse IP}
}
% Auteur du document (étudiant-e) en projet de Bachelor
\author{Maxime Scharwath}

% Activer l'option pour l'accord du féminin dans le texte
\genre{male}

% Titre de votre travail de Bachelor
\title{Edition collaborative de documents structurés avec un chiffrement de bout-en-bout et un stockage distribué}

% Le sous titre est optionnel
\subtitle{Travail de Bachelor}

% Nom du professeur responsable
\teacher {Prof. Chapuis Bertil (HEIG-VD)}

% Mettre à jour avec la date de rendu du travail
\date{\today}

% Numéro de TB
\thesis{7212}



\surroundwithmdframed{minted}

%% Début du document
\begin{document}
\selectlanguage{french}
\maketitle
\frontmatter
\clearemptydoublepage

%% Requis par les dispositions générales des travaux de Bachelor
\preamble
\authentification

%% Résumé / Résumé publiable / Version abrégée
\begin{abstract}
    % Francais
\lipsum[1]

\asterism

% English
\lipsum[3]

\end{abstract}

%% Sommaire et tables
\clearemptydoublepage
{
    \tableofcontents
    \let\cleardoublepage\clearpage
    \listoffigures
    \let\cleardoublepage\clearpage
    \listoftables
    \let\cleardoublepage\clearpage
    \listoflistings
}

\printnomenclature
\clearemptydoublepage
\pagenumbering{arabic}

%% Contenu
\mainmatter
\chapter{Introduction}
\section{Contexte et introduction}

Le développement de solutions pour la synchronisation et la réplication de données distribuées est un défi majeur dans le domaine de l'informatique moderne. S'attaquant à ce défi, Condensation, une startup innovante, a créé une gamme de logiciels dédiés à garantir la cohérence des données à travers un réseau de n\oe{}uds. Leur système de synchronisation sans conflit assure l'intégrité des données et une sécurité de bout en bout.

L'architecture initiale de Condensation a été conçue par \textbf{Thomas Lochmatter} et est actuellement maintenue et développée par la startup elle-même. L'enjeu de la synchronisation des données distribuées rend ce projet particulièrement stimulant, notamment en raison des limites des solutions existantes.

Initialement, ce travail de Bachelor visait à collaborer avec Condensation pour développer un démonstrateur utilisant leur système. Cet outil aurait été conçu pour mettre en valeur les fonctionnalités du système et fournir aux développeurs des exemples concrets pour faciliter l'intégration de la bibliothèque dans leurs projets. Cependant, en raison d'une restructuration interne à Condensation, notamment leur décision de réécrire leur système en Rust, cette collaboration n'a pas pu se concrétiser comme prévu.

Face à cette situation, la direction de ce projet de Bachelor a dû être révisée. Au lieu d'utiliser le système existant de Condensation, il a fallu développer de A à Z un système de synchronisation et de réplication de données distribuées. Ce changement d'orientation a entraîné une augmentation significative du travail nécessaire pour ce projet. Cependant, cela a aussi créé une opportunité unique d'apprentissage et de développement de compétences techniques avancées.

En plus du développement de ce nouveau système, une application de création de whiteboard a été conçue pour illustrer son utilisation pratique. Le défi de créer à la fois le système et une application utilisant ce système a été une expérience très stimulante. Je suis particulièrement fier de ces réalisations et je suis enthousiaste à l'idée de les présenter dans ce rapport.

Bien que la collaboration directe avec Condensation n'ait pas eu lieu comme prévu, les principes et l'approche de Condensation ont eu une influence significative sur ce travail. Ce rapport décrira donc les différentes étapes de ce projet réorienté, les choix techniques réalisés, ainsi que les défis rencontrés et les solutions apportées.
%%if
\section{Exemple d'équation}
L'une des principales forces de \LaTeX~est la saisie d'équations. L'équation \ref{eq:1}, citée à titre d'exemple, représente la transformation de phase d'une lentille biconvexe. Pour rédiger une équation \LaTeX~vous pouvez utiliser des outils en ligne tels que \href{https://www.latex4technics.com/}{latex4technics}. Essayez autant que possible d'écrire vos équations à la main. La courbe d'apprentissage n'est pas très raide et la valeur ajoutée est grande. Vous pouvez vous aider du panneau de \LaTeX~Workshop dans Visual Studio Code. Il est accessible via le raccourcis clavier \keystroke{Ctrl} + \keystroke{Alt} + \keystroke{X}.

\begin{equation} \label{eq:1}
    \begin{split}
        L(x,y) &= \exp\left( - i\frac{{2\pi }}{\lambda }\left( {n\Delta \varphi (x,y) + \Delta {\varphi _0} - \Delta \varphi (x,y)} \right)\right)\\
        &= {\exp\left({i\frac{{2\pi }}{\lambda }\Delta {\varphi _0}}\right)}{\exp\left({ - i\frac{{2\pi }}{{\lambda f}}({x^2} + {y^2})}\right)}
    \end{split}
\end{equation}

\section{Exemples de diagrammes}

Les diagrammes de flux peuvent être réalisés en utilisant l'outil \href{https://app.diagrams.net/}{draw.io}. Une exportation en \texttt{.drawio} (non compressé) permet de garder les sources de la figure. Le rendu en \texttt{.pdf} sera réalisé à la volée à la compilation. L'intérêt est double : n'avoir qu'une source de vérité \cad pas d'image intermédiaire à stocker, et réduire la quantité d'information stockée.

Puisque la source est au format XML, les textes sont accessibles au correcteur orthographique et il vous est rendu possible les modifier sans avoir à éditer l'image. La figure \ref{euclide.drawio} en est un exemple.

Notons qu'il est inutile d'insérer des images coloriées là où la couleur n'offre aucune valeur ajoutée ; évitez également les ombrages et autres effets de style. Enfin, préférez toujours des représentations vectorielles là où c'est possible.

Ce modèle apporte la commande \verb!\fig! qui peut prendre plusieurs options. Utilisez \verb!H! pour forcer la figure à apparaître à l'endroit de la déclaration. Ajustez la largeur de la figure à \SI{80}{\percent} de largeur de page avec \verb!width=0.8\textwidth!.

\section{Exemple de figure}

Pour présenter vos résultats d'expérience, vous pouvez soit dessiner des graphiques manuellement en utilisant des outils de dessin vectoriel comme Inkscape ou Adobe Illustrator, comme illustré à la figure \ref{plot.svg}.

\fig[H, width=0.8\textwidth]{Exemple de graphique plan}{plot.svg}

Vous pouvez utiliser Python ou Matlab pour générer des figures à la volée à partir d'une source de données. À titre d'exemple, le code source \ref{python} permet de générer la figure \ref{bode.py}.
\begin{listing}[h]
    \inputminted{python}{assets/figures/bode.py}
    \caption{Génération d'un diagramme de Bode \label{python}}
\end{listing}

\fig[H, width=12cm]{Diagramme de Bode généré à la volée}{bode.py}

\subsection{Example de schéma électronique}
Vous pouvez également utiliser TikZ pour créer vos propres schémas électriques et électroniques comme l'exemple \ref{circuit}. N'hésitez pas à vous inspirer d'exemples disponibles sur internet (\href{https://texample.net/tikz/examples/area/electrical-engineering/}{texample/electrical-engineering}).

\begin{figure}[h]
    \begin{center}
        \begin{circuitikz}
            \draw
            (0,0) to [short, *-] (6,0)
            to [V, l_=$\mathrm{j}{\omega}_m \underline{\phi}^s_R$] (6,2)
            to [R, l_=$R_R$] (6,4)
            to [short, i_=$\underline{i}^s_R$] (5,4)
            (0,0) to [open, v^>=$\underline{u}^s_s$] (0,4)
            to [short, *- ,i=$\underline{i}^s_s$] (1,4)
            to [R, l=$R_s$] (3,4)
            to [L, l=$L_{\sigma}$] (5,4)
            to [short, i_=$\underline{i}^s_M$] (5,3)
            to [L, l_=$L_M$] (5,0);
        \end{circuitikz}
        \caption{Circuit électrique \label{circuit}}
    \end{center}
\end{figure}

\subsection{Dessins techniques}
La présentation de dessins mécaniques est préférée en vue filaire. SolidWorks conserve la représentation vectorielle à l'exportation mais pas lorsqu'il y a des textures ou des rendus. À partir du PDF généré, l'image peut être isolée et sauvegardée en format SVG.

\begin{figure}[!ht]
    \begin{center}
        \includegraphics[width=10cm]{\assetsdir/assembly.svg.pdf}
    \end{center}
    \caption[Assemblage mécanique]{\label{assembly}Réducteur cycloïdale de puissance comportant 6. l'axe de sortie, 14. le roulement de sortie, 1. le corps du réducteur en aluminium, 3 et 5. les disques cycloïdaux et 2. les goupilles de prise... D'autres informations liées à la figure elle-même peuvent aussi figurer dans la légende}
\end{figure}

Notez ici que la légende est particulièrement longue. Celle que vous retrouverez dans la table figures est plus courte. La commande \mintinline{latex}{\caption[courte]{longue}} permet de saisir une légende courte pour la table des figures et une légende longue pour documenter la figure. Utilisez \mintinline{latex}{\fig[short=Légende courte]{Légende longue}{fichier}}.

La figure \ref{assembly} est un dessin technique épuré qui permet de décrire un phénomène ou un fonctionnement important dans le rapport technique. Les mises en plan détaillées seront quant à elles disponibles en annexes.

\section{Tableaux}

Concernant les tableaux un seul conseil : restez simple et minimaliste, n'ajoutez des séparateurs que là ou c'est nécessaire pour améliorer la lisibilité. Une liste de quelques cantons suisses est donnée à titre d'exemple dans la table \ref{cantons}.

\begin{table}[h]
    \begin{center}
        \caption{Liste des cantons \label{cantons}}
        \begin{tabular}{c|l|r}
            Abréviation & Nom du canton & Depuis                  \\ \hline
            ZH          & Zürich        & \ordinalnum{1} mai 1351 \\
            BE          & Berne         & 6 mars 1353             \\
            FR          & Fribourg      & 22 décembre 1481        \\
            VD          & Vaud          & 19 février 1815         \\
            VS          & Valais        & 4 août 1815             \\
            NE          & Neuchâtel     & 19 mai 1815             \\
            GE          & Genève        & 19 mai 1815
        \end{tabular}
    \end{center}
\end{table}

Comparez la lisibilté de cette même table avec celle que vous pourriez trouver dans un document Word :

\begin{table}[h]
    \begin{center}
        \caption{Liste des cantons (vilain)}
        \begin{tabular}{|l|l|l|} \hline
            \textbf{Abréviation} & \textbf{Nom du canton} & \textbf{Depuis}         \\
            \Xhline{4\arrayrulewidth}
            ZH                   & Zürich                 & \ordinalnum{1} mai 1351 \\ \hline
            BE                   & Berne                  & 6 mars 1353             \\ \hline
            FR                   & Fribourg               & 22 décembre 1481        \\ \hline
            VD                   & Vaud                   & 19 février 1815         \\ \hline
            VS                   & Valais                 & 4 août 1815             \\ \hline
            NE                   & Neuchâtel              & 19 mai 1815             \\ \hline
            GE                   & Genève                 & 19 mai 1815             \\ \hline
        \end{tabular}
    \end{center}
\end{table}

Si vous devez donner une spécification technique, n'oubliez pas de mentionner les valeurs minimales, maximales et nominales sans omettre l'unité de mesure. Notez que les séparateurs verticaux sont souvent critiqués pour réduire la lisibilité mais parfois ils sont utiles. Utilisez-les avec parcimonie. Jouez avec l'alignment des colonnes pour accroître la lisibilité et utilisez l'environmement \mintinline{latex}{tabularx} pour plus d'unité dans les largeurs de vos tableaux.

\begin{table}[h]
    \begin{center}
        \caption{Exigences techniques \label{specification}}
        \begin{tabularx}{\textwidth}{cXcccr}
            \toprule
            No. & Exigence                                                                   & Min. & Nom. & Max. & Unité                           \\
            \midrule
            E1  & Tension d'alimentation                                                     & 12   & 24   & 48   & \si{\volt}                      \\
            E2  & Fréquence                                                                  & 50   &      & 60   & \si{\hertz}                     \\
            E3  & Concentration                                                              &      & 300  & 1200 & \si{\nano\gram\per\milli\litre} \\
            E4  & \multicolumn{5}{l}{Doit pouvoir être stoppé à l'aide d'un arrêt d'urgence}                                                        \\
            \bottomrule
        \end{tabularx}
    \end{center}
\end{table}

L'exemple de la table \ref{specification}, assigne pour chaque exigence un numéro unique. Cette table est \textbf{normative}, chaque élément doit pouvoir être référencé par un identifiant unique (cf. T\ref{specification}-E3). Dans le cas ou cet identifiant est utilisé en dehors de ce document, la version du document devra être renseignée.

\section{Index}
\LaTeX~ permet d'indexer les mots \index{mots} importants. Il suffit de placer les termes importants d'un paragraphe dans la commande \mintinline{latex}{\index{terme}} et ils apparaîtront automatiquement à la fin de ce rapport dans l'index du document.

\index{Napoléon}

Imaginons que dans cette section nous parlions du cheval blanc \index{cheval blanc} de Napoléon. Il se pourrait que le lecteur recherche ce passage dans la version imprimée du rapport. Avec l'index, rien de plus facile. Allez jeter un oeil à la page \pageref{index}.

\section{Notes de bas de page}

\maraja{Je suis une marginale, et je suis utile pour résumé un paragraphe en quelques mots.} Parfois, il est plus élégant d'annoter une définition en utilisant une note de bas de page \footnote{La note en bas de page (ou note de bas de page) est une forme littéraire, consistant en une ou plusieurs lignes ne figurant pas dans le texte.}. Alternativement il est possible d'annoter un paragraphe avec une note marginale.

\section{Glossaire et acronymes}

La \Gls{heig-vd} membre de la \Gls{hes-so} propose ce modèle de document. Le format \LaTeX~est particulièrement adapté pour les documents qui contiennent des expressions mathématiques. Pour plus de détail sur l'utilisation d'un glossaire, se référer à \href{https://www.overleaf.com/learn/latex/Glossaries}{Overleaf/Glossaires}. Tient donc, ci-dessus nous utilisons deux acronymes. Les trouverez-vous dans le glossaire en page \pageref{glossaire} ?

\section{Unités de mesure}

Lorsque vous mentionnez des quantités, utilisez les unités du système international. \LaTeX~et le paquet \texttt{siunitx} permet la saisie de quantités. La commande suivante permet d'afficher \SI{42.12}{\kilo\gram\metre\per\square\second}.\par

\mintinline{latex}{\SI{42.12}{\kilo\gram\metre\per\square\second}}\par

Notez qu'une espace fine précède l'unité et que ces dernières ne sont pas en italiques.
%%fi

\chapter{Conclusion}
\section{État final du projet}

Le projet Describble s'est achevé avec succès, résultant en la création d'un tableau blanc décentralisé. Cette application offre un éventail complet de fonctionnalités de dessin, gère efficacement les documents, et est dotée d'une interface multilingue. De plus, elle a été déployée en production et répond à toutes les exigences spécifiées dans le cahier des charges révisé.

\section{Fonctionnalités additionnelles}

Au-delà des objectifs initiaux, Describble a incorporé des fonctionnalités supplémentaires qui améliorent considérablement l'expérience utilisateur. La présence en temps réel offre la possibilité de voir les curseurs des autres utilisateurs, renforçant ainsi le caractère interactif de l'application. La capacité de travailler hors ligne offre plus de flexibilité à l'utilisateur, permettant l'accès et la modification des documents même sans connexion internet. Une gestion avancée des accès aux documents a été mise en place, offrant à l'utilisateur un contrôle accru sur qui peut accéder à ses documents. En outre, l'application a été optimisée pour une utilisation sur appareils mobiles et est compatible avec les Progressive Web Applications (PWA), permettant ainsi une accessibilité et une utilisation plus larges. Enfin, l'intégration de l'internationalisation rend l'application accessible à un public plus large, transcendant les barrières linguistiques.

\section{Changements et défis rencontrés}

Initialement prévue pour une collaboration avec Condensation, cette collaboration a dû être réévaluée en raison d'une restructuration au sein de Condensation. Cela a exigé le développement d'un système de synchronisation et de réplication de données distribuées à partir de zéro, augmentant ainsi la complexité du projet. Cependant, ce défi a également permis d'acquérir des compétences techniques avancées et de bénéficier d'une expérience d'apprentissage précieuse.

\section{Apprentissages réalisés et vécu du travail de Bachelor}

Au cours de ce travail de Bachelor, j'ai acquis des compétences techniques significatives, en particulier dans le domaine de la synchronisation et de la réplication de données distribuées. La nécessité de réorienter le projet a offert une occasion d'apprendre à naviguer efficacement à travers les imprévus et à gérer les changements de direction. Malgré les défis supplémentaires engendrés par cette réorientation, ces obstacles ont été surmontés avec succès.

La réalisation de ce travail de Bachelor a été une expérience enrichissante et stimulante. L'accomplissement final, en dépit des défis rencontrés, a été d'autant plus gratifiant. Cela a non seulement renforcé mes compétences techniques, mais m'a aussi permis de gagner en résilience et en adaptabilité face aux imprévus.

\section{Remerciements}

Je tiens à exprimer ma gratitude à mon superviseur, le Professeur Bertil Chapuis. Sa guidance et son soutien tout au long de ce projet ont été précieux. Je suis également reconnaissant pour son aide dans la facilitation de la communication avec la direction de l'HEIG-VD et pour l'obtention du temps plein alloué pour ce travail de Bachelor.

J'aimerais également remercier l'équipe de Condensation. Même si notre collaboration n'a pas pu se concrétiser comme prévu, j'ai beaucoup appris grâce à eux sur le monde des startups et sur la décentralisation. Leur travail a eu une influence significative sur ce projet et m'a offert une perspective précieuse sur le potentiel de la décentralisation.

Enfin, je tiens à remercier ma famille et mes amis pour leur encouragement constant tout au long de ce parcours. Leur soutien a été une source constante d'inspiration et de motivation.

\hspace{8cm}\makeatletter\@author\makeatother\par
\hspace{8cm}\begin{minipage}{5cm}
\end{minipage}
\printsignature

\clearpage
\printbibliography

\appendix
\appendixpage
\addappheadtotoc

%%if
\chapter{Première annexe}

Les annexes n'ont pas un contenu \underline{normatif} mais \underline{descriptif}. Tout contenu annexé ne doit pas être nécessaire à la bonne compréhension du travail.

Les annexes contiennent généralement :

\begin{itemize}
    \item les dessins mécaniques (mises en plan);
    \item les schémas électriques détaillés;
    \item des photographies du projet;
    \item des scripts et des extraits de code source;
    \item des documents techniques \pex \emph{datasheet};
    \item des développements mathématiques.
\end{itemize}
\section{Sous section}
\lipsum[1]
%%fi

\let\cleardoublepage\clearpage
\backmatter

\label{glossaire}
\printnoidxglossary
\label{index}
\printindex

% Le colophon est le dernier élément d'un document qui contient des notes de l'auteur concernant la mise en page et l'édition du document : il est parfaitement optionnel.
%%if
\clearpage
\Large\textbf{Colophon :}\par\normalsize
\thispagestyle{empty}
La qualité de cet ouvrage repose que le moteur \LaTeX. La mise en page et le format sont inspirés d'ouvrages scientifiques tels que le modèle de thèse de l'EPFL et celui des publications O'Reilly.

Les diagrammes et les illustrations sont édités depuis l'outil en ligne draw.io. Certaines illustrations ont été reprises dans Adobe Illustrator. Les représentations 3D sont exportées de SolidWorks et certains graphiques sont générés à la volée depuis un code source Python.

L'auteur fictive de ce document \emph{Maria Bernasconi} est un nom emprunté, par amusement, aux spécimens publiés par Postfinance.

Ce document a été compilé avec XeLaTeX.

La famille de police de caractères utilisée est \emph{Computed Modern} créée par Donald Knuth avec son logiciel METAFONT.
\vfil
Le Colophon est le dernier élément d'un document qui contient des notes de l'auteur concernant la mise en page et l'édition du document : il est parfaitement optionnel.
%%fi

\end{document}
