\section{Définitions}
\subsection{Document structuré}
Un document structuré organise l'information de manière hiérarchique avec des éléments et attributs spécifiques. Il facilite la compréhension, la manipulation et la réutilisation des informations, contrairement aux documents non structurés.

Dans un tableau blanc collaboratif, un document structuré représente les éléments graphiques et textuels. Il aide à manipuler et modifier les éléments individuels, gérer l'historique des modifications et synchroniser les données entre les participants.

Le format SVG qui utilise le langage XML est un exemple de document structuré. Il est utilisé pour décrire la structure et le contenu d'un document vectoriel. Il est composé d'éléments graphiques, tels que des rectangles, des cercles et des lignes, et de textes.

\begin{listing}[H]
    \inputminted{XML}{assets/code/svg-example.svg}
    \caption{Exemple de document SVG \label{fig:svg-example}}
\end{listing}

\subsection{Édition collaborative de documents}
L'édition collaborative permet à plusieurs utilisateurs de travailler simultanément sur un document en fusionnant les modifications en temps réel.
Plusieurs outils existent pour l'édition collaborative de documents structurés, un exemple connu est Google Docs.

\section{Méthodes et protocoles pour resoudre les conflits de modification}
Lors de l'édition collaborative de documents structurés, les utilisateurs
peuvent modifier le document en même temps. Les modifications peuvent être
effectuées sur des éléments différents ou sur le même élément. Dans ce cas, il y
a un conflit de modification. Les conflits de modification peuvent être résolus
de différentes manières, par exemple en utilisant un mécanisme de verrouillage
ou en utilisant un algorithme de fusion. Dans les sections suivantes, nous
allons aborder les concepts clés, les mécanismes et les systèmes basés sur ces méthodes.

\subsection{Operational Transformation (OT)}
La Transformation Opérationnelle (OT) est un mécanisme de synchronisation de données en temps réel qui résout les conflits entre modifications concurrentes apportées par les utilisateurs. L'OT repose sur la transformation des opérations pour préserver leur intention lorsqu'elles sont appliquées dans un ordre différent. Dans cette section, nous aborderons les concepts clés, les mécanismes et les systèmes basés sur l'OT.

\subsubsection{Concepts de base de l'OT}
Pour comprendre le fonctionnement de l'OT, il est important de se familiariser avec les concepts de base suivants:

\paragraph{Opération:} Une opération est une action effectuée par un utilisateur sur un document, telle que l'insertion ou la suppression d'un caractère. Les opérations sont généralement représentées par des objets contenant des informations sur l'action effectuée, la position dans le document et, le cas échéant, le caractère inséré ou supprimé.
\paragraph{Intention:} L'intention d'une opération est l'effet souhaité de l'opération sur le document. L'OT vise à préserver l'intention des opérations, même lorsqu'elles sont appliquées dans un ordre différent.
\paragraph{Conflit:} Un conflit se produit lorsque deux opérations sont effectuées simultanément sur le même document, et que l'application de ces opérations dans un ordre différent peut entraîner des résultats incohérents. L'OT résout les conflits en transformant les opérations concurrentes de manière à préserver leur intention.

\subsubsection{Transformation des opérations et Conditions de transformation}
La transformation des opérations (OT) permet de préserver l'intention des opérations concurrentes en les modifiant. Supposons que deux utilisateurs, Alice et Bob, travaillent sur un document. Les opérations d'Alice et de Bob sont notées $O_A$ et $O_B$ respectivement. L'OT utilise les fonctions de transformation $T_1$ et $T_2$ pour résoudre les conflits, tel que :

\begin{equation}
    \begin{aligned}
        O_A' = T_1(O_A, O_B) \\
        O_B' = T_2(O_B, O_A)
    \end{aligned}
\end{equation}

Où $O_A'$ et $O_B'$ sont les opérations transformées. L'application de $O_A'$ et $O_B'$ sur le document préserve l'intention et résout les conflits.

Pour garantir que les opérations transformées sont cohérentes, peu importe l'ordre d'application, l'algorithme OT définit des propriétés de transformation (TP). Ces conditions, parfois appelées commutativité et idempotence, peuvent s'écrire :

\begin{equation}
    \begin{aligned}
        T_1(O_A', O_B') = O_A \\
        T_2(O_B', O_A') = O_B
    \end{aligned}
\end{equation}

Cela signifie que les opérations initiales $O_A$ et $O_B$ sont retrouvées en transformant à nouveau les opérations transformées $O_A'$ et $O_B'$ l'une par rapport à l'autre.

En définissant ces conditions de transformation appropriées, l'algorithme OT préserve les intentions des utilisateurs et garantit un document final cohérent, même en cas de travail simultané sur un même document.

\subsubsection{Algorithmes basés sur l'OT}
Plusieurs algorithmes et systèmes basés sur l'OT ont été développés, notamment Jupiter et Google Wave.

Jupiter est un système de collaboration en temps réel OT, créé au début des années 1990 au Xerox PARC. Il fonctionne en client-serveur, chaque client et serveur possédant sa propre copie du document et appliquant les opérations localement.

Google Wave était un système de collaboration en temps réel conçu par Google, basé sur l'OT. Il utilisait un modèle décentralisé où chaque utilisateur maintenait sa propre copie du document, offrant plus de flexibilité pour gérer les modifications hors ligne et synchroniser les données lorsqu'un utilisateur se reconnectait.

Cependant, Google Wave ne parvint pas à attirer suffisamment d'utilisateurs et son développement fut arrêté. Néanmoins, certaines idées et techniques furent intégrées dans d'autres produits de collaboration en temps réel de Google, tels que Google Docs.

\subsubsection{Limitations et défis de l'OT}
Bien que l'OT soit un mécanisme puissant pour la synchronisation en temps réel des documents et la résolution des conflits, il présente également certaines limitations et défis. Par exemple, les algorithmes basés sur l'OT peuvent être complexes à mettre en oeuvre et à déboguer, particulièrement lorsqu'ils doivent gérer des cas de conflits complexes et des relations de causalité. De plus, la transformation des opérations et le contrôle de la causalité peuvent avoir un impact sur les performances, surtout lorsque le nombre d'utilisateurs et d'opérations concurrentes augmente. Malgré ces défis, l'OT reste une méthode importante et largement utilisée pour la collaboration en temps réel et la synchronisation des données dans les systèmes distribués.

\subsection{Conflict-Free Replicated Data Types (CRDT)}
Les Conflict-Free Replicated Data Types (CRDT) sont des structures de données distribuées qui facilitent la collaboration en temps réel sur des données partagées sans nécessiter un serveur central. Les opérations sur les CRDT sont commutatives et idempotentes, assurant ainsi une convergence stable et une cohérence des données dans un environnement distribué.

\subsubsection{Principes des CRDT}
Les CRDT sont distribués et répliqués sur plusieurs noeuds, chaque noeud possédant une copie des données. Les noeuds peuvent modifier les données localement et les synchroniser entre eux. Chaque opération est associée à un identifiant unique pour garantir la commutativité et l'idempotence des opérations.

\subsubsection{Exemple d'application des CRDT : Logoot}
Pour illustrer le fonctionnement des CRDT, considérons le cas de Logoot, un type de CRDT conçu pour la collaboration en temps réel sur des documents textuels. Dans l'algorithme Logoot, chaque caractère dans le document est associé à un identifiant unique, qui est une paire d'éléments notée sous la forme `(x, y)`. Le premier élément `x` correspond à la position du caractère dans le document, tandis que le second élément `y` est un identifiant unique généré lors de l'insertion du caractère.

Prenons l'exemple de deux utilisateurs, Alice et Bob, collaborant sur un document texte initialisé avec le mot "chat".

\paragraph{Scénario}
Alice et Bob travaillent sur un texte initial "chat". Alice veut former "chou" en ajoutant "ou" après "ch", pendant que Bob veut former "chef" en ajoutant "ef" après "ch".

\paragraph{Logoot CRDT}
Nous illustrons ci-dessous comment Alice et Bob peuvent collaborer sur le document texte en utilisant Logoot CRDT :

Le document initial est :

\begin{equation}
    c(1),\ h(2), \ a(3), t(4) \quad \text{(chat)}
\end{equation}

Ici, chaque caractère du document est suivi d'une paire d'éléments `(x, y)`, où `x` est la position du caractère dans le document et `y` est un identifiant unique généré lors de l'insertion du caractère.

\paragraph{Opérations d'insertion}
Alice insère "ou" après "ch" et génère de nouvelles positions :

\begin{equation}
    o(2, 1), \ u(2, 2)
\end{equation}

Le texte d'Alice devient :

\begin{equation}
    c(1), \ h(2), \ o(2, 1), \ u(2, 2), \ a(3), \ t(4) \quad \text{(chouat)}
\end{equation}

Bob insère "ef" après "ch" et génère aussi de nouvelles positions :

\begin{equation}
    e(2, 3), f(2, 4)
\end{equation}

Le texte de Bob devient :

\begin{equation}
    c(1), \ h(2), \ e(2, 3), \ f(2, 4), \ a(3), \ t(4) \quad \text{(chefat)}
\end{equation}

\paragraph{Intégration des opérations}
Les opérations d'insertion sont ensuite échangées. Alice intègre l'insertion de Bob, et Bob intègre l'insertion d'Alice. Le résultat est identique pour Alice et Bob, malgré un ordre différent d'opérations. Les CRDT assurent convergence et cohérence sans transformation opérationnelle.

\subsubsection{Applications des CRDT}
Les CRDT inspirent de nombreux algorithmes et systèmes de collaboration en temps réel, tels que WOOT, Treedoc, RGA, Logoot, LSEQ, Automerge et Yjs.

\paragraph{Automerge\cite{hardenbergAutomerge2023}: } Automerge est une bibliothèque JavaScript pour développer des applications collaboratives en temps réel, comme des éditeurs de texte et des tableaux Kanban. Elle utilise des CRDT pour synchroniser les données entre les instances de l'application, en maintenant la cohérence et en résolvant les conflits.

\paragraph{Yjs\cite{nicolaescuYjsFrameworkRealTime2015}: } Yjs est une bibliothèque JavaScript de collaboration en temps réel qui utilise aussi les CRDT pour la synchronisation des données. Yjs supporte diverses structures de données et fonctionnalités, dont la collaboration sur des documents texte, des tableaux et des graphes.

En résumé, les CRDT sont une approche puissante pour la collaboration en temps réel. Elles résolvent les problèmes de synchronisation et de cohérence des données sans recourir à la transformation opérationnelle. Cependant, elles ne sont pas adaptées à tous les types de données et structures de données, et leur mise en oeuvre peut être complexe.

\section{Zéro Trust}
Le modèle Zéro Trust est une approche de cybersécurité basée sur l'idée de ne faire confiance à aucune entité, interne ou externe, dans un système informatique. On vérifie constamment les identités et les autorisations avant d'accorder l'accès aux ressources. Cette méthode réduit les risques de failles de sécurité et protège les données sensibles contre les accès non autorisés.

\section{Chiffrement de bout en bout}

Le chiffrement de bout en bout (E2EE) est un système de communication qui garantit que seuls les utilisateurs participant à la conversation peuvent lire les messages échangés. Il a pour but de protéger la confidentialité des communications en empêchant les tiers - tels que les fournisseurs de télécommunications, les acteurs malveillants, voire même les fournisseurs de services - d'accéder aux clés de chiffrement nécessaires pour déchiffrer les messages. En conséquence, seuls l'expéditeur et les destinataires légitimes sont capables de lire ou modifier les données échangées. L'expéditeur chiffre les messages, tandis que le tiers ne dispose pas des moyens pour les déchiffrer, les stockant donc sous forme chiffrée. Les destinataires récupèrent les données chiffrées et les déchiffrent eux-mêmes, assurant ainsi la confidentialité de la communication.

\subsection{Fonctionnement du chiffrement de bout en bout}

Dans sa forme la plus simple, le chiffrement de bout en bout utilise des clés de chiffrement asymétriques pour chaque utilisateur participant à une communication sécurisée. Chaque utilisateur possède une paire de clés, une publique et une privée. La clé publique est partagée librement, tandis que la clé privée reste secrète et accessible uniquement à l'utilisateur concerné.

Pour envoyer un message chiffré, un utilisateur A utilise la clé publique de l'utilisateur B pour chiffrer le message. Une fois chiffré, seul B peut déchiffrer le message avec sa clé privée. Ce processus est également appelé chiffrement asymétrique.

Illustrons ceci avec une application de messagerie chiffrée de bout en bout. Alice utilise la clé publique de Bob pour chiffrer un message. Le message chiffré est envoyé et stocké sur le serveur de messagerie. Même si le serveur est compromis ou que les données sont interceptées, le message reste illisible sans la clé privée de Bob. Bob déchiffre ensuite le message avec sa clé privée.

Le partage des clés de chiffrement doit être effectué avec prudence pour éviter leur interception. Les clés doivent être échangées de manière sécurisée, en utilisant un canal de communication sûr.

Il existe de nombreuses variantes à cette approche de base, permettant de renforcer la sécurité et de gérer des situations plus complexes. Par exemple, certaines méthodes de chiffrement de bout en bout offrent une sécurité "backward" et "forward". La sécurité backward signifie que si une clé privée est compromise, seuls les messages futurs sont en danger - les messages passés restent sécurisés. La sécurité forward garantit que si une clé privée est compromise, seuls les messages passés sont en danger - les messages futurs restent sécurisés.

De plus, dans certains systèmes, une combinaison de chiffrement asymétrique et symétrique est utilisée pour bénéficier à la fois de la sécurité du chiffrement asymétrique et de l'efficacité du chiffrement symétrique. Par exemple, un secret partagé peut être établi à l'aide de l'échange de clés Diffie-Hellman sur courbe elliptique (ECDH), puis ce secret est utilisé pour le chiffrement et le déchiffrement des messages à l'aide de l'Advanced Encryption Standard (AES).

Ces méthodes avancées de chiffrement de bout en bout permettent de répondre à divers besoins et de s'adapter à un large éventail de situations, tout en assurant un haut niveau de sécurité des communications.

\section{Conclusion}
En conclusion, l'état de l'art dans le domaine de l'édition collaborative de
documents, du chiffrement de bout en bout et du stockage distribué montre qu'il
existe plusieurs approches pour aborder ces problématiques. Les systèmes basés
sur resolution de conflits, tels que Jupiter et Google Wave, ont apporté des
contributions significatives dans la synchronisation des données en temps réel
et la résolution des conflits. Cependant, des défis subsistent, tels que la
complexité et les performances des algorithmes.

Face à ces défis, les chercheurs et les développeurs continuent d'explorer de nouvelles approches et d'améliorer les technologies existantes pour répondre aux besoins croissants en matière de collaboration, de sécurité et de décentralisation des données.