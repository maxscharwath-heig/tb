\section{État final du projet}

Le projet s'est conclu avec la création réussie de l'application Describble, un tableau blanc décentralisé qui offre une gamme complète de fonctionnalités de dessin, la gestion des documents, l'internationalisation et le déploiement en production. L'application répond à toutes les exigences définies dans le cahier des charges révisé.

\section{Fonctionnalités supplémentaires}

En dépassant les attentes initiales, Describble a intégré des fonctionnalités supplémentaires, notamment la présence en temps réel qui permet de visualiser les curseurs des autres utilisateurs, la possibilité de travailler hors ligne et une gestion avancée des accès aux documents.

\section{Changements et défis rencontrés}

Malgré l'anticipation initiale d'une collaboration avec Condensation, ce plan a dû être modifié en raison d'une restructuration interne à Condensation. Il a fallu développer un système de synchronisation et de réplication de données distribuées à partir de zéro, augmentant la complexité du projet. Cependant, ce défi a également offert une occasion précieuse d'apprendre et d'acquérir des compétences techniques avancées.

\section{Apprentissages réalisés}

Ce travail de Bachelor a permis de gagner en compétence technique, notamment dans le domaine de la synchronisation et de la réplication de données distribuées. En plus de ces compétences techniques, la réorientation du projet a été une occasion d'apprendre à naviguer efficacement à travers les imprévus et à gérer les changements de direction.

\section{Vécu du travail de Bachelor}

La réalisation de ce travail de Bachelor a été une expérience enrichissante et stimulante. Les défis supplémentaires engendrés par la réorientation initiale du projet ont été surmontés avec succès, rendant l'accomplissement final d'autant plus gratifiant.

\section{Remerciements}

Je tiens à exprimer ma gratitude à mon superviseur, le Professeur Bertil Chapuis. Sa guidance et son soutien tout au long de ce projet ont été précieux. Son aide pour faciliter la communication avec la direction de l'HEIG-VD et pour obtenir le temps plein alloué pour ce travail de Bachelor a été particulièrement appréciée.

Je voudrais également remercier l'équipe de Condensation. Bien que notre collaboration n'ait pas pu se concrétiser comme prévu, j'ai beaucoup appris sur le monde des startups et sur le sujet de la décentralisation grâce à eux. Leurs travaux ont eu une influence significative sur ce projet et m'ont offert une perspective précieuse sur le potentiel de la décentralisation.

Enfin, un grand merci à ma famille et mes amis pour leur encouragement constant tout au long de ce parcours. Leur soutien a été une source constante d'inspiration et de motivation.

\hspace{8cm}\makeatletter\@author\makeatother\par
\hspace{8cm}\begin{minipage}{5cm}
\end{minipage}
\printsignature