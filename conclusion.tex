\section{État final du projet}

Le projet \gls{Describble} s'est achevé avec succès, résultant en la création d'un tableau blanc décentralisé. Cette application offre un éventail complet de fonctionnalités de dessin, gère efficacement les documents, et est dotée d'une interface multilingue. De plus, elle a été déployée en production et répond à toutes les exigences spécifiées dans le cahier des charges révisé.

\section{Fonctionnalités additionnelles}

Au-delà des objectifs initiaux, \gls{Describble} a incorporé des fonctionnalités supplémentaires qui améliorent considérablement l'expérience utilisateur. La présence en temps réel offre la possibilité de voir les curseurs des autres utilisateurs, renforçant ainsi le caractère interactif de l'application. La capacité de travailler hors ligne offre plus de flexibilité à l'utilisateur, permettant l'accès et la modification des documents même sans connexion internet. Une gestion avancée des accès aux documents a été mise en place, offrant à l'utilisateur un contrôle accru sur qui peut accéder à ses documents. En outre, l'application a été optimisée pour une utilisation sur appareils mobiles et est compatible avec les Progressive Web Applications (\gls{PWA}), permettant ainsi une accessibilité et une utilisation plus larges. Enfin, l'intégration de l'internationalisation rend l'application accessible à un public plus large, transcendant les barrières linguistiques.

\section{Changements et défis rencontrés}

Initialement prévue pour une collaboration avec Condensation, cette collaboration a dû être réévaluée en raison d'une restructuration au sein de Condensation. Cela a exigé le développement d'un système de synchronisation et de réplication de données distribuées à partir de zéro, augmentant ainsi la complexité du projet. Cependant, ce défi a également permis d'acquérir des compétences techniques avancées et de bénéficier d'une expérience d'apprentissage précieuse.

\section{Apprentissages et vécu du travail de Bachelor}

Au cours de ce travail de Bachelor, j'ai acquis des compétences techniques significatives, en particulier dans le domaine de la synchronisation et de la réplication de données distribuées. La nécessité de réorienter le projet a offert une occasion d'apprendre à naviguer efficacement à travers les imprévus et à gérer les changements de direction. Malgré les défis supplémentaires engendrés par cette réorientation, ces obstacles ont été surmontés avec succès.

La réalisation de ce travail de Bachelor a été une expérience enrichissante et stimulante. L'accomplissement de ce projet, en dépit des enjeux rencontrés, a été d'autant plus gratifiant. Cela a non seulement renforcé mes compétences techniques, mais m'a aussi permis de gagner en résilience et en adaptabilité face aux imprévus.

\section{Remerciements}

Je tiens à exprimer ma gratitude à mon superviseur, le Professeur Bertil Chapuis. Sa guidance et son soutien tout au long de ce projet ont été précieux. Je suis également reconnaissant pour son aide dans la facilitation de la communication avec la direction de l'HEIG-VD et pour l'obtention du temps plein alloué pour ce travail de Bachelor.

J'aimerais encore remercier l'équipe de Condensation. Même si notre collaboration n'a pas pu se concrétiser comme prévu, j'ai beaucoup appris grâce à eux sur le monde des startups et sur la décentralisation. Leur travail a eu une influence significative sur ce projet et m'a offert une perspective précieuse sur le potentiel de la décentralisation.

Enfin, je tiens à remercier ma famille et mes amis pour leur encouragement tout au long de ce parcours. Leur soutien a été une source constante d'inspiration et de motivation.

\hspace{8cm}\makeatletter\@author\makeatother\par
\hspace{8cm}\begin{minipage}{5cm}
\end{minipage}
\printsignature